\begin{UseCase}{CU14}{Generar Multas}{
		Generar las multas a las cuales los usuarios serán acreedores cuando no se cumple con las normas que se establecidas y aceptadas al solicitar la credencial correspondiente para el préstamo de ejemplares de la biblioteca, las cuales corresponden a: pérdida, devolución a desatiempo, maltrato o sustitución de un libro.
	}
		\UCitem{Versión}{1.0}
		\UCitem{Actor}{Administrador}
		\UCitem{Propósito}{Generar multas a los usuarios del servicio de la biblioteca que no cumplan con las condiciones de devolución.}
		\UCitem{Entradas}{
			\begin{itemize}
				\item Tipo de afectación en el ejemplar.
				\item Identificador del ejemplar.
				\item Identificador del usuario.
			\end{itemize}					
		}
		\UCitem{Salidas}{
			\begin{itemize}
				\item Documento de penalización con los siguientes datos: nombre de usuario, nombre de libro, tipo de multa y cantidad acreedora.
			\end{itemize}					
		}
		\UCitem{Precondiciones}{ 
			\begin{itemize}
				\item El usuario debe de estar registrado. 
				\item El usuario debe de tener un ejemplar registrado como "Prestamo".
				\item El administrador verifica el estado del libro y del préstamo en el CU9.
			\end{itemize}
		}
		\UCitem{Postcondiciones}{
			\begin{itemize}
				\item Se registrará una multa en el historial del usuario.
			\end{itemize}					
		}
		\UCitem{Autor}{
			\begin{itemize}
				\item José Miguel Tejeda Martínez
				\item Erubey Martínez de los Santos					
			\end{itemize}	
		}
		\UCitem{Revisores}{
			\begin{itemize}
				\item López Orozco Diego Efraín
				\item Jiménez Chávez Luis Gerardo					
			\end{itemize}	
		}
		\UCitem{Madurez}{
			\begin{itemize}
				\item Media
			\end{itemize}	
		}
		
		\UCitem{Prioridad}{
			\begin{itemize}
				\item Alta
			\end{itemize}	
		}
		
		\UCitem{Estado}{
			\begin{itemize}
				\item Pendiente
			\end{itemize}	
		}
\end{UseCase}
\begin{UCtrayectoria}{Principal}
	\UCpaso[\UCactor] El administrador introduce el tipo de multa que busca aplicar al usuario.
	\UCpaso[\UCactor] El administrador selecciona el botón \IUbutton{Generar} de la interfaz: \IUref{UI14}{Generar Multa}. 
	\UCpaso[\UCsist]  El sistema verifica el tipo de multa introducido, de acuerdo a las reglas de negocio \BRref{RN11}{RN11}, \BRref{RN12}{RN12}, \BRref{RN13}{RN13}, \BRref{RN14}{RN14}.\Trayref{A}
	\UCpaso[\UCsist] El sistema obtiene el costo del libro de a cuerdo al identificador proporcionado y se muestra en la interfaz \IUref{UI14}{Generar Multa}. \Trayref{B}
	\UCpaso[\UCsist] El sistema obtiene las fechas de préstamo y de devolución establecidas en la emisión anterior y se muestra en la interfaz \IUref{UI14}{Generar Multa}. \Trayref{B}
	\UCpaso[\UCsist] El sistema calcula la multa de acuerdo a las siguientes condiciones, correspondientes al tipo de multa: \BRitem
	1. "Pérdida de libro". Se considerará el costo total obtenido en el paso 2. \BRitem
	2. "Devolución a destiempo". Se tomará la fecha de devolución del paso 3 y contarán los días dde diferencia entre la fecha actual. La cantidad de días multiplicará por 20. \BRitem
	3. "Maltrato de libro". Se cosiderará el costo en el paso 2 y se multiplicara por 0.35 \BRitem
	4. "Sustitución de libro". Se considerará el costo obtenido en el paso 2 y se multiplicará por 0.5.
	\UCpaso[\UCsist] El sistema descargara el DOC-1, en el que devolverá la cantidad de multa correspondiente al paso anterior, la cual se asociará al usuario dado por el identificador de entrada. \Trayref{B} 
	\UCpaso[\UCsist] Se mostrará el mensaje MSG1
\end{UCtrayectoria}
\begin{UCtrayectoriaA}{A}{Se introdujo un número de multa inexistente}	
			\UCpaso[\UCsist] El sistema mostrará el mensaje [E2].
			\UCpaso[\UCsist] Regresa al paso 1 de la trayectoria principal. 
\end{UCtrayectoriaA}
\begin{UCtrayectoriaA}{B}{Se presentó un error en la conexión y/o en la transacción actual con base de datos}
			\UCpaso[\UCsist] El sistema mostrará el mensaeje [E1].
			\UCpaso[\UCsist] Regresa al paso 1 de la trayectoria principal.
\end{UCtrayectoriaA}
