\begin{UseCase}{CU14}{Generar Multas}{
		Generar las multas a las cuales los usuarios serán acreedores por no cumplir las normas establecidas y aceptadas al obtener su credencial para poder obtener ejemplares de la biblioteca.
	}
		\UCitem{Versión}{1.3}
		\UCitem{Actor}{Administrador}
		\UCitem{Propósito}{Generar multas a los usuarios del servicio de la biblioteca que no cumplan con las condiciones de devolución.}
		\UCitem{Entradas}{
			\begin{itemize}
				\item Tipo de afectación en el ejemplar.
				\item Identificador del ejemplar.
				\item Identificador del usuario.
			\end{itemize}					
		}
		\UCitem{Salidas}{
			\begin{itemize}
				\item Documento con la informacion relacionada a la multa generada, conteniendo: usuario, libro afectado, tipo de multa y monto.
			\end{itemize}					
		}
		\UCitem{Precondiciones}{ 
			\begin{itemize}
				\item El usuario debe de estar registrado. 
				\item El usuario debe de tener un ejemplar registrado como "Prestamo".
				\item El administrador verifica el estado del libro y del préstamo en el CU9.
			\end{itemize}
		}
		\UCitem{Postcondiciones}{
			\begin{itemize}
				\item Se registrará una multa en el historial del usuario.
			\end{itemize}					
		}
		\UCitem{Autor}{
				José Miguel Tejeda Martínez
				Erubey Martínez de los Santos
		}
		\UCitem{Revisores}{
			Luis Gerardo Jimenez Chavez
			Diego Efraín López Orozco
		}
		\UCitem{Estado}{
			Aprobado
		}
\end{UseCase}
\begin{UCtrayectoria}{Principal}
	\UCpaso[\UCactor] El administrador introduce el tipo de multa aplicable.
	\UCpaso[\UCactor] El administrador introduce el identificador del usuario que será sancionado.
	\UCpaso[\UCactor] El administrador introduce el identificador del libro afectado.
	\UCpaso[\UCactor] El administrador selecciona el botón \IUbutton{Generar} de la interfaz: \IUref{UI14}{Generar Multa}.
	\UCpaso[\UCsist] El sistema verifica los tipos de multa y datos introducidos de acuerdo a las reglas de negocio \BRref{RN11}{Perdida de libro}, \BRref{RN11}{Devolución a desatiempo}, \BRref{RN11}{Maltrato de libro}, \BRref{RN11}{Sustitución de libro}.\Trayref{A}
	\UCpaso[\UCsist] El sistema obtiene el costo del libro de a cuerdo al identificador proporcionado y se muestra en la interfaz \IUref{UI14}{Generar Multa}. \Trayref{B}
	\UCpaso[\UCsist] El sistema obtiene las fechas de préstamo y de devolución establecidas en la emisión anterior y se muestra en la interfaz \IUref{UI14}{Generar Multa}. \Trayref{B}
	\UCpaso[\UCsist] El sistema calcula la multa de acuerdo a las siguientes condiciones, correspondientes al tipo de multa: \BRitem
	1. "Pérdida de libro". Se considerará el costo total obtenido en el paso 2. \BRitem
	2. "Devolución a destiempo". Se tomará la fecha de devolución del paso 3 y contarán los días dde diferencia entre la fecha actual. La cantidad de días multiplicará por 20. \BRitem
	3. "Maltrato de libro". Se cosiderará el 35 por ciento del costo en el paso 2. \BRitem
	4. "Sustitución de libro". Se considerará el 50 por ciento del costo obtenido en el paso 2.
	\UCpaso[\UCsist] El sistema descargara el DOC-1, en el que devolverá la cantidad de multa correspondiente al paso anterior, la cual se asociará al usuario dado por el identificador de entrada. \Trayref{B} 
	\UCpaso[\UCsist] Se mostrará el \MSGref{MSG1}{Operación exitosa}.
\end{UCtrayectoria}
\begin{UCtrayectoriaA}{A}{Se introdujo un número de multa inexistente}	
			\UCpaso[\UCsist] El sistema mostrará el mensaje \MSGref{[E2]}{Datos incorrectos}.
			\UCpaso[\UCsist] Regresa al paso 1 de la trayectoria principal. 
\end{UCtrayectoriaA}
\begin{UCtrayectoriaA}{B}{Se presentó un error en la conexión y/o en la transacción actual con base de datos}
			\UCpaso[\UCsist] El sistema mostrará el mensaeje \MSGref{[E2]}{Error de conexión con BD}.
			\UCpaso[\UCsist] Regresa al paso 1 de la trayectoria principal.
\end{UCtrayectoriaA}