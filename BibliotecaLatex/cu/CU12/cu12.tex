% Copie este bloque por cada caso de uso:
%-------------------------------------- COMIENZA descripción del caso de uso.

	\begin{UseCase}{CU12}{Apartar Libro}{
		El objetivo es describir la funcionalidad del sistema donde se podrá apartar un libro cuando este no este disponible, para que un libro no este disponible quiere decir que otro Usuario lo tiene, el apartado genera una orden para que el pueda recogerlo al estar disponible nuevamente y solo puede existir una orden para un libro, el libro debe existir en la biblioteca para ser apartado, para que de esta manera el usuario que aparte tenga la seguridad de que al llegar el libro a la biblioteca pueda ir a recogerlo y hacer uso de el. Siempre y cuando el usuario este registrado. Puede haber mas de un libro para poder ser prestado pero solo se considera el primer caso cuando no existe ninguno disponible. Pueden extistir libros en la biblioteca pero no pueden ser prestados como Libros Viejos, Libros Maltratados, etc. El usuario registrado debe estar limpio de multas y penalizaciones. El usuario debera asistir a la biblioteca a recoger el libro cuando este este disponible. El sistema envia un correo cuando tenga 1 libro disponible.
	}
		\UCitem{Versión}{0.2}
		\UCitem{Actor}{Alumno}
		\UCitem{Responsables}{Guillermo Eder López Rojas \newline Rodrigo Burciaga}
		\UCitem{Propósito}{Generar una orden de apartado de libro para obtener un libro al volver a estar disponible. }
		\UCitem{Entradas}{
		\begin{itemize}
			\item Nombre del Usuario
			\item Id del Usuario
			\item Fecha del Día Actual
			\item Nombre de Libro Requerido
		\end{itemize}		
		}
		\UCitem{Salidas}{Orden de Apartado de Libro. Que contendrá: id de orden de apartado, fecha del día en que se hizo la orden, nombre de quien hizo la orden, nombre del libro, autor, id del libro, medios de contacto del alumno.}
		
		\UCitem{Precondiciones}{ 
			\begin{itemize}
				\item Interna: Que exista el libro. 
				\item Interna: Que el libro no esté disponible.
				\item Interna: Que el libro no esté apartado ya.
				\item Interna: Que el usuario este registrado en la base de datos.
				\item Interna: Que exista catálogo de ordenes de apartado.
			\end{itemize}
		}
		\UCitem{Postcondiciones}{Se actualiza la tabla LibrosApartados de la base de datos con la orden de apartado correspondiente.}
		\UCitem{Tipo}{Primario. Extiende el caso de uso \UCref{5}}
		\UCitem{Autor}{López Orozco Diego Efraín \newline Luis Gerardo Jimenez}
	\end{UseCase}
		%-------------------------------------- COMIENZA descripción Trayectoria Principal
	\begin{UCtrayectoria}{Principal}
		\UCpaso[\UCactor] Busca el libro dando clic en la opción de \IUbutton{Buscar Libro} que aparece en la \IUref{UI1201}{Pantalla de Inicio}.
		\UCpaso[\UCsist] Llama al caso de uso \UCref{5}.
		\UCpaso[\UCsist] Muestra la intefaz para iniciar el proceso de apartar libro \IUref{UI1202}{Libro no Disponible}.
		\UCpaso[\UCactor] Da clic en el boton \IUbutton{Apartar Libro} que aparece en la \IUref{UI1202}{Libro no Disponible}.
		\UCpaso[\UCsist] Verifica si el libro no esta disponible conforme a la regla de negocio \BRref{RN20}{Libro Disponible para Apartado}\Trayref{A}.
		\UCpaso[\UCsist] Despliega la ventana \IUref{UI1203}{Formulario Apartar Libro} mostrando la interfaz de usuario.
		\UCpaso[\UCactor] Realiza el llenado del campo: Clave del Usuario (id).
		\UCpaso[\UCactor] Da clic en el botón de \IUbutton{Apartar} en la parte inferior derecha de la \IUref{UI1203}{Formulario Apartar Libro}.
		\UCpaso[\UCsist] Realizará la validación de que no haya campos vacios conforme a la regla de negocio \BRref{RN42}{Datos del formulario completos}\Trayref{B}.		
		\UCpaso[\UCsist] Realizará la validación de los datos conforme a las reglas de negocio \BRref{RN18}{Apartado Libro} \BRref{RN19}{Apartado para Usuario}\Trayref{C}\Trayref{D}.
		
		\UCpaso[\UCsist] Llenará la tabla de la entidad LibrosApartados en la base de datos \Trayref{E}, donde se guardarán los registros nuevos de los libros apartados (Orden de apartado). 
		\UCpaso[\UCsist] Enviará un correo con los datos de la orden de apartado y los datos del libro \Trayref{F}.		
		\UCpaso[\UCsist] Muestra el \MSGref{MSG1}{Operación Exitosa},
	\end{UCtrayectoria}
			%-------------------------------------- COMIENZA descripción Trayectoria Alternativa.
		\begin{UCtrayectoriaA}{A}{El libro se encuentra disponible}
			\UCpaso[\UCsist] Muestra el \MSGref{MSG5}{Libro Disponible}. Notificando la violación de la \BRref{RN20}{Libro Disponible para Apartado}.
			\UCpaso[\UCactor] Oprime el botón \IUbutton{Ok}.
			\UCpaso[\UCsist] Regresa al paso 3 de la trayectoria principal.
		\end{UCtrayectoriaA}		
		
		\begin{UCtrayectoriaA}{B}{Existen campos vacios}
			\UCpaso[\UCsist] Muestra el \MSGref{E3}{Datos incompletos. Favor de completar todos os campos de entrada}. Notificando la violación de la \BRref{RN42}{Datos de formulario completos}.
			\UCpaso[\UCactor] Oprime el botón \IUbutton{Ok}.
			\UCpaso[\UCsist] Regresa al paso 6 de la trayectoria principal.
		\end{UCtrayectoriaA}
		
		\begin{UCtrayectoriaA}{C}{El libro ya fue apartado}
			\UCpaso[\UCsist] Muestra el \MSGref{MSG3}{Libro Apartado}. Notificando la violación de la \BRref{RN18}{Apartado Libro}.
			\UCpaso[\UCactor] Oprime el botón \IUbutton{Ok}.
			\UCpaso[\UCsist] Regresa al paso 3 de la trayectoria principal.
		\end{UCtrayectoriaA}
		
		\begin{UCtrayectoriaA}{D}{El usuario no se encuentra registrado}
			\UCpaso[\UCsist] Muestra el \MSGref{E4}{Usuario sin Registrar}. Notificando la violación de la \BRref{RN19}{Apartado para Usuario}.
			\UCpaso[\UCactor] Oprime el botón \IUbutton{Ok}.
			\UCpaso[\UCsist] Regresa al paso 6 de la trayectoria principal.
		\end{UCtrayectoriaA}
		
		
		\begin{UCtrayectoriaA}{E}{No se realizo exitosamente la conexión a la base de datos}
			\UCpaso[\UCsist] Muestra el \MSGref{E1}{Sin conexión a la BD}. Notificando que no se pudo realizar la conexión a la base de datos.
			\UCpaso[\UCactor] Oprime el botón \IUbutton{Ok}.
			\UCpaso[\UCsist] Regresa al paso 6 de la trayectoria principal.
		\end{UCtrayectoriaA}
		
		\begin{UCtrayectoriaA}{F}{El email no se puedo enviar}
			\UCpaso[\UCsist] Muestra el \MSGref{E6}{Libro Apartado}. Notificando que no se pudo enviar el correo al usuario.
			\UCpaso[\UCactor] Oprime el botón \IUbutton{Ok}.
			\UCpaso[\UCsist] Regresa al paso 12 de la trayectoria principal.
		\end{UCtrayectoriaA}
		
%-------------------------------------- TERMINA descripción del caso de uso.
