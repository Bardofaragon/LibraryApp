% Copie este bloque por cada caso de uso:
%-------------------------------------- COMIENZA descripción del caso de uso.

	\begin{UseCase}{CU13}{Registrar Multas}{
	Las multas de los usuarios se registrarán para obligar al usuario a cumplir con la sanción.
	}
		\UCitem{Versión}{0.4}
		\UCitem{Actor}{Bibliotecario}
		\UCitem{Propósito}{El bibliotecario registrará las multas de los usuarios para obligarlos a cumplir su sanción, mostrando el tipo de multa,la fecha en la que fue registrada,monto de la multa y las multas registradas con anterioridad. }
		\UCitem{Entradas}{
						\begin{itemize}
        					\item Nombre de usuario: Cadena de caracteres [Requerido]
        					\item Tipo de multa: Cadena de caracteres [Requerido]
        					\begin{itemize}
        						\item Pérdida de ejemplar
        						\item Devolución de ejemplar a destiempo
        						\item Maltrato de ejemplar (Romper, rayar, pintar, deshojar)
        						\item Sustitución de ejemplar
        					\end{itemize} 
	        				\item Descripción: Cadena de texto
        				\end{itemize} 
		}
		\UCitem{Salidas}{
				\begin{itemize}
	        		\item Comprobante de multa: Archivo PDF que contiene:
	        		\begin{itemize}
	        			\item Información de multas:
	        			\begin{itemize}
	        				\item Nombre del usuario: Cadena de texto
	        				\item Fecha de multa: Fecha en formato dd/mm/yyyy
			        		\item Tipo de multa: Cadena de texto
	    		    		\item Descripción: Cadena de texto
	        				\item Sanción o Monto: Cadena de texto
			        		\item Multas atrasadas: Cadena de texto
	    		    		\item Fecha de límite de cumplimiento: Fecha en formato dd/mm/yyyy
	        			\end{itemize}
	     		   \end{itemize}
	        	   \item Mensaje de confirmación
	        	   \begin{itemize}
	        			\item Información de multa:
	        			\begin{itemize}
	        				\item Nombre del usuario: Cadena de texto
	        				\item Tipo de multa: Cadena de texto
		        			\item Descripción: Cadena de texto
		        			\item Monto: Cadena de texto
	    	    			\item Multas atrasadas: Cadena de texto
	        			\end{itemize}
	        		\end{itemize}
	    		\end{itemize}		
		}
		\UCitem{Precondiciones}{ 
			\begin{itemize}
            		\item El usuario debe estar registrado en el sistema
            		\item El usuario debe ser acreedor a una multa, \BRref{RN33}{Registro de multa}
            		\item El usuario no debe encontrarse en estado bloqueado
       		 \end{itemize}
		}
		\UCitem{Postcondiciones}{
			\begin{itemize}
	    			\item El estado del usuario cambia de no-multado o multado a multado o bloqueado. 
	    	\end{itemize}			
		}
		\UCitem{Autor}{Martínez Vilchis Juan Moisés y Perea Mejía Luis.}
		\UCitem{Reviso}{Mujica Marquez Victor Edgar, Guarneros Sanatana Victor Hugo}
		\UCitem{Estado}{Aprbado}
	\end{UseCase}
%-------------------------------------- COMIENZA descripción Trayectoria Principal
	\begin{UCtrayectoria}{Principal}
		\UCpaso[\UCactor] Solicita registrar una multa haciendo click sobre el botón \IUbutton{Registrar Multa} en la pantalla principal IUX.
		\UCpaso Muestra la pantalla \IUref{IU13A}{RegistrarMulta}.
		\UCpaso[\UCactor] Proporciona el nombre del usuario multado ingresándolo desde el teclado.
		\UCpaso[\UCactor] Proporciona el tipo de multa seleccionándolo desde la lista que se muestra en pantalla.
		\UCpaso[\UCactor] Proporciona la descripción de la multa ingres\'andolo desde el teclado.
		\UCpaso[\UCactor] Confirma operación haciendo click sobre el botón \IUbutton{Continuar} en la pantalla\IUref{IU13A}{RegistrarMulta}
		\UCpaso Verifica que el usuario esté registrado en el sistema.\Trayref{A}
		\UCpaso Verifica que se hayan completado los datos de entrada requeridos con el formato correcto.\Trayref{B}
		\UCpaso Obtiene los datos del usuario acreedor a la multa.
		\UCpaso Guarda la información de la multa correspondiente con el estado de Pendiente	
		\UCpaso Asigna un monto a la multa correspondiendo con el tipo de multa. \BRref{RN51}{Monto por Pérdida} \BRref{RN52}{Monto por entrega a destiempo} \BRref{RN53}{Monto por maltrato} \BRref{RN54}{Monto por sustitución} 	
		\UCpaso Cambia el estado del usuario a Multado \Trayref{C}
		\UCpaso Confirma operación mostrando mensaje de confirmación, datos de multa  en la pantalla \IUref{IU13B}{Confirmacion}, y se envía un archivo PDF del comprobante  al correo electrónico del usuario \Trayref{D}
		\UCpaso[\UCactor] Confirma operación haciendo click sobre el botón OK.
		\UCpaso Muestra la pantalla IUX.	
\end{UCtrayectoria}

			%-------------------------------------- COMIENZA descripción Trayectoria Alternativa.
\begin{UCtrayectoriaA}{A}{Usuario no registrado,
Cuando se haya violado la regla \BRref{RN04}{Usuario no registrado}}
	\UCpaso Informa al actor mostrando el error \MSGref{E4}{Registro no encontrado} 
	\UCpaso Continúa trayectoria desde paso 2 
\end{UCtrayectoriaA}
			
		\begin{UCtrayectoriaA}{B}{Datos incorrectos,
Cuando se haya violado la regla \BRref{RN37}{Formato del nombre}}
	\UCpaso Informa al actor mostrando el error \MSGref{E2}{Datos introducidos incorrectamente} 
	\UCpaso Continúa trayectoria desde paso 2 
\end{UCtrayectoriaA}


\begin{UCtrayectoriaA}{C}{Cambio de estado,
Cuando se haya violado la regla \BRref{RN07}{Usuario bloqueado}}
	\UCpaso Cambia el estado del usuario a Bloqueado mostrando el mensaje  \MSGref{MSG4}{El usuario se encuentra bloqueado}
	\UCpaso Continúa trayectoria desde paso 14 
\end{UCtrayectoriaA}

\begin{UCtrayectoriaA}{D}{Error de conexión,
Cuando se haya violado la regla \BRref{RN24}{Registro completo}}
	\UCpaso Informa al actor mostrando el error  \MSGref{E1}{Error en la conexión con la base de datos}
	\UCpaso Continúa trayectoria desde paso 2 
\end{UCtrayectoriaA}


%-------------------------------------- TERMINA descripción del caso de uso.


\newpage
%%%%%%%% pantalla principal
\subsection{DS13 Diagrama de secuencia}
Diagrama de secuencia:
\\
El presente diagrama tiene como finalidad servir de apoyo al programador para que este pueda tener claras las clases y métodos que deberán ser programadas.
\IUfig[.7]{CU13/DCU13}{DS13}{Diagrama de Secuencia}
