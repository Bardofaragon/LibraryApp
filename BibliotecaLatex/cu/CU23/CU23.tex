% Copie este bloque por cada caso de uso:
%-------------------------------------- COMIENZA descripción del caso de uso.

	\begin{UseCase}{CU23}{Notificación de préstamo interbibliotecario}{
		El Bibliotecario podrá consultar todas la solicitudes de préstamo que ha echo para saber cuales ya han sido aceptadas y cuales no, esto lo hace mostrando una tabla y el estado en que se encuentra la petición Aceptado, Pendiente o Rechazado
	}
		\UCitem{Versión}{0.1}
		\UCitem{Actor}{Bibliotecario}
		\UCitem{Propósito}{Visualizar el estado en que se encuentre la solicitad de libro.}
		\UCitem{Entradas}{Ninguna}
		\UCitem{Salidas}{Titulo, Autor, Editorial,Año de publicación, No. de paginas ,Biblioteca,Ejemplares, Estado.Mensaje \MSGref{E1}{Error al conectar a la base de datos.}}
		\UCitem{Precondiciones}{Ninguna}
		\UCitem{Postcondiciones}{Ninguna}
		\UCitem{Autor}{Rodriguez Cervantes Arturo y Monserrat Ceron Rodriguez .}
		\UCitem{Revisó}{Diego Manriquez Molina y Abiran Natanael Salas Hernandez}
	\end{UseCase}
		%-------------------------------------- COMIENZA descripción Trayectoria Principal
	\begin{UCtrayectoria}{Principal}
		\UCpaso[\UCactor] Del CU08 presiona el botón:  \IUbutton{Notificación préstamo interbiblitecario} de la pantalla \IUref{UI08.2}{préstamo interbibliotecario}.
		\UCpaso[\UCsist]Muestra la pantalla \IUref{UI23.1}{Notificación de préstamo interbibliotecario}. Con todas las solicitudes de préstamo que se han echo con el estado de cada una
		\UCpaso[\UCactor]Visualiza los datos en la pantalla.
	\end{UCtrayectoria}
			%-------------------------------------- COMIENZA descripción Trayectoria Alternativa.
		\begin{UCtrayectoriaA}{A}{Error al conectar a la BD}
			\UCpaso[\UCsist] Muestra el mensaje \MSGref{E1}{Error al conectar a la base de datos.}
			\UCpaso[\UCactor] Presiona el botón \IUbutton{OK}
			\UCpaso[\UCsist] Regresa al paso 2 de la trayectoria principal.
		\end{UCtrayectoriaA}
TERMINA descripción del caso de uso.
