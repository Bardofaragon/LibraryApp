% Copie este bloque por cada caso de uso:
%-------------------------------------- COMIENZA descripción del caso de uso.

	\begin{UseCase}{CU09}{Registrar Devolución de libro}{
		El bibliotecario registrará la devolución de un libro que fue previamente prestado con base a al ISBN del libro(Identificador del libro) e identificador del usuario al que se le fue prestado dicho libro, al mismo tiempo se actualizará y verificará el estado del libro devuelto refiriendose a la disponibilidad y al estado fisico.
	}
		\UCitem{Versión}{0.3}
		\UCitem{Actor}{Bibliotecario, Usuario}
		\UCitem{Propósito}{Registrar y verificar en el sistema la devolución de un libro, con base al nombre del usuario e identificador del libro actulizando las disponibilidad del mismo para su prestamo en un futuro, con el fin de tener un control sobre el inventario de libros de la biblioteca.}
		\UCitem{Entradas}{
		Se necesitan de los siguientes datos
					\begin{itemize}
				\item Identificador del libro 
				\item Nombre del usuario que devuelve el libro
				\item Fecha en la que se devuelve el libro
			\end{itemize}}
		\UCitem{Salidas}{
								\begin{itemize}
				\item Devolución registraada
				\item Se actuliza el estado de disponibilidad del libro
				\item Se actualiza los libros que el usuario adeuda
				\item Si es el caso se aplicará una multa
			\end{itemize}
		
		}
				\UCitem{Precondiciones}{ 
			\begin{itemize}
				\item Interna: Debe estar iniciada una sesión como bibliotecario
				\item Interna: Que exista al menos un Bibliotecario en la base de datos 
				\item Interna: Que exista al menos un Usuario registrado en la base de datos
				\item Interna: Que exista al menos 2 libros registrados y prestados con su respectiva fecha limite para devolver el libro
			\end{itemize}
		}
		\UCitem{Postcondiciones}{El estado de disponibilidad del libro sera actualizado, al igual que los libros del usuario tiene pendientes de devolver y se registrará la devolución del libro dentro de la base de datos.}
		\UCitem{Autor}{Esteban Pablo Martínez Ibáñez}
		\UCitem{Status}{En revisión}
			\UCitem{Responsable de revisión}{BAUTISTA ROSALES MAURICIO}
	\end{UseCase}
		%-------------------------------------- COMIENZA descripción Trayectoria Principal
	\begin{UCtrayectoria}{Principal}
		\UCpaso[\UCactor] El usuario se identifica y solicita devolver un libro en prestamo al bibliotecario.  
		\UCpaso[\UCactor] El bibliotecario ingresará el nombre del usuario que devuelve el libro de acuerdo con el formato de la regla de negocio \BRref{RN37}{Formato del nombre}, el identificador del libro que se va a devolver y la fecha en la que el libro es entregado \BRref{RN48}{Formato de la fecha.}\Trayref{A}\Trayref{B} \IUref{UI0109}{Pantalla de verificación} 
		\UCpaso[\UCactor] El bibliotecario dara click al Boton \IUbutton{Verificar}.
				\UCpaso[\UCsist] Obtiene los datos ingresados y muestra la información correspondiente como: el usuario, libro a devolver, fecha en la que se devuelve, fecha limite a devolver dicho libro y la seccion del estado fisico en el que se devuelve el libro \IUref{UI0209}{Pantalla de información}.
\UCpaso[\UCactor] El bibliotecario seleeccionará la casilla indicando las condiciones fisicas del libro devuelto.						
						\UCpaso[\UCactor] El bibliotecario dara click al Boton \IUbutton{Registrar Devolución} .
		\UCpaso[\UCsist] Se verificara la fecha en la que el usuario devuelve con la fecha limite para devolver el libro con base en la \BRref{RN10}{Devolucion en fecha} y las condiciones en las que se encuentra dicho libro. \Trayref{C}
				\UCpaso[\UCsist] Se actualiza el registro del estado del libro asi como su Disponibilidad.  \IUref{UI0309}{Pantalla de información} \Trayref{D}
	\UCpaso[\UCactor] El bibliotecario dara click al Boton \IUbutton{OK}.
	\end{UCtrayectoria}
			%-------------------------------------- COMIENZA descripción Trayectoria Alternativa.
				\begin{UCtrayectoriaA}{A}{El bibliotecario ingresa el nombre con el formato incorrecto}
			\UCpaso[\UCsist] Se mostrará un mensaje de advertencia notificando que los datos ingresados no son correctos. \MSGref{MASG6}{Ingrese los datos correctamente}
			\UCpaso[\UCactor] El bibliotecario volvera a ingresar el nombre de manera correcta conforme a la Regla de negocio \BRref{RN37}{Formato del nombre} 
			\UCpaso[\UCsist]Continuar en el paso 4 de la trayectoria Principal
		\end{UCtrayectoriaA}
		
		
\begin{UCtrayectoriaA}{B}{El bibliotecario ingresa la fecha con el formato incorrecto}
			\UCpaso[\UCsist] Se mostrará un mensaje de advertencia notificando que los datos ingresados no son correctos. \MSGref{MASG6}{Ingrese los datos correctamente}
			\UCpaso[\UCactor] El bibliotecario volvera a ingresar la fecha de devolución del libro de manera correcta conforme a la Regla de negocio \BRref{RN48}{Formato de la fecha}
			\UCpaso[\UCsist]Continuar en el paso 4 de la trayectoria Principal 
		\end{UCtrayectoriaA}




		\begin{UCtrayectoriaA}{C}{El usuario devuelve un libro depues de la fecha especificada}
			\UCpaso[\UCactor] El bibliotecario verifica la fecha de devolución.
			\UCpaso[\UCsist] Se mostrará un mensaje de advertencia notificando que la fecha de devolución ha vencido. \MSGref{MSG10}{Se ha vencido la fecha de entrega}
			\UCpaso[\UCsist] Se Seguira el proceso correspondiente con respecto al caso de uso 14 CU14 para aplicar una multa correspondiente
		\end{UCtrayectoriaA}


		\begin{UCtrayectoriaA}{D}{El sistema no registra la devolución del libro}
			\UCpaso[\UCsist] Se mostrará un mensaje de advertencia notificando el problema. \MSGref{MSG11}{El registro no se realizó, intentelo de nuevo}
			\UCpaso[\UCactor]El bibliotecario repite el paso 6 de la trayectoria Principal.
		\end{UCtrayectoriaA}
%-------------------------------------- TERMINA descripción del caso de uso.
