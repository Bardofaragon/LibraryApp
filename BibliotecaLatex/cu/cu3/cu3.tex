% Copie este bloque por cada caso de uso:
%-------------------------------------- COMIENZA descripción del caso de uso.

	\begin{UseCase}{CU03}{Generar Credencial}{
		Generar una credencial en formato PDF para un usuario dado, pidiendo sus datos por medio de un 			formulario que recibe: 
		\begin{itemize}
			\item Nombre 
			\item Apellido paterno
			\item Apellido materno
			\item Teléfono
			\item Fecha de emisión y la Fecha de expedición
			\item Tipo de usuario
		\end{itemize}				
		
	}
		\UCitem{Versión}{0.2}
		\UCitem{Actor}{Administrador}
		\UCitem{Propósito}{Brindar al usuario una credencial que lo reconozca como miembro de la biblioteca para que pueda hacer uso de sus servicios. }
		\UCitem{Entradas}{Usuario:
		\begin{itemize}
			\item Nombre [String]
			\item Apellido Paterno [String]
			\item Apellido Materno [String]
			\item Teléfono [String]
			\item Tipo de Usuario [String] 
			\item Fecha de Emision [Date] (dd/mm/aaaa)
			\item Fecha de Expiración [Date] (dd/mm/aaaa)
		\end{itemize}
}
		\UCitem{Salidas}{El usuario obtiene su credencial}
		\UCitem{Precondiciones}{ 
			\begin{itemize}
				\item El aspirante debe ya estar registrado en el sistema
				\item El aspirante no deberá tener adeudos
			\end{itemize}
		}
		\UCitem{Postcondiciones}{
			\begin{itemize}
				\item Se guardarán los datos de la credencial en la base de datos.
				\item El usuario podrá hacer uso de los servicios de la biblioteca.
			\end{itemize} 
			}
		\UCitem{Autores}{Hernández Flores Luis Ángel y Hernandez Aguirre Ezequiel Absalon}
		\UCitem{Revisó}{Durán MArtínez Josué y Castañeda Sánchez Miguel Ángel}
	\end{UseCase}
	
		%-------------------------------------- COMIENZA descripción Trayectoria Principal
	\begin{UCtrayectoria}{Principal}
	\UCpaso[\UCactor] Selecciona el botón:  \IUbutton{Generar Credencial} de la pantalla IGU1 \IUref{UI01}{Pantalla de Inicio}.	
		\UCpaso[\UCsist] Se conecta a la BD \Trayref{A}.
		\UCpaso[\UCsist] Valida las reglas de negocio RN2 y RN15 \Trayref{B}.
		\UCpaso[\UCsist] Produce la credencial del usuario
		\UCpaso[\UCsist] Envía al usuario el mensaje \MSGref{MSG01}{Operación realizada exitosamente}. \Trayref{C}.
		\UCpaso[\UCactor]Da click en \IUbutton{Aceptar}.

		\end{UCtrayectoria}		
		%	-------------------------------------- COMIENZA descripción Trayectoria Alternativa.
		\begin{UCtrayectoriaA}{A}{No se puede conectar a la Base de datos}
			\UCpaso[\UCsist] Envía al usuario un mensaje \MSGref{E1}{Error al conectar la Base de datos}
			\UCpaso[\UCsist] El administrador da click en el botón \IUbutton{Aceptar}.
		\end{UCtrayectoriaA}
		
		\begin{UCtrayectoriaA}{B}{Campos nulos}
			\UCpaso[\UCsist] Envía al usuario un mensaje \MSGref{E3}{Los campos no pueden estar vacios}
			\UCpaso[\UCsist] El administrador da click en el botón \IUbutton{Aceptar}.
		\end{UCtrayectoriaA}
		
		





%-------------------------------------- TERMINA descripción del caso de uso.