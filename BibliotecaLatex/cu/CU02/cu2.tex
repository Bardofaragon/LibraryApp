% Copie este bloque por cada caso de uso:
%-------------------------------------- COMIENZA descripción del caso de uso.

	\begin{UseCase}{CU02}{Generar cuentas de usuario}{
	   Se le pedirá al administrador de las cuentas de usuario que ingrese los datos solicitados, a partir de éste momento se comenzará con la  validación de los mismos; una vez realizado esto, se almacenará la instancia de usuario final en la base de datos "Se generará su cuenta" para poder acceder a los servicios ofrecidos en la biblioteca, los usuarios finales serán de varios tipos dependiendo del rol que se le asigne.
	}
		\UCitem{Versión}{1}
		\UCitem{Actor}{Administrador de Cuentas de Usuario}
		\UCitem{Propósito}{Generar nuevas a los usuarios finales, para que puedan acceder a los servicios ofrecidos por la biblioteca (consultar procesos de negocio). Se hace énfasis, los servicios a los que se tendrá acceso por usuario dependerán en función del rol asignado a cada cuenta de usuario.}
		\UCitem{Entradas}{
		\begin{itemize}
		\item Una instancia de la clase Usuario
		  \begin{itemize}
		   \item nombres[String]
		   \item apellidoPa[String]
		   \item apellidoMa[String]
		   \item email[String]
		   \item calle[String]
		   \item numCalle[String]
		   \item cp[int]
		   \item curp[String]
		   \item telefono[String]
		   \item rol[int]
		   \item foto[file]
		  \end{itemize}
		\end{itemize}				
		}
		\UCitem{Salidas}{
		\begin{itemize}
		\item Se enviará un correo al usuario con sus datos de registro y su idUsuario que se generará automaticamente
		\item Actualizará la capa de modelo ''insertar en la base de datos el registro''	
		\end{itemize}				
		}
		\UCitem{Precondiciones}{ 
			\begin{itemize}
				\item El administrador debe haber ingresado al sistema mediante el 			                 enlace. 
				\item El usuario no tuvo que haber sido registrado anteriormente
				\item Deberá existir el catálogo para usuarios
			\end{itemize}
		}
		\UCitem{Postcondiciones}{
		\begin{itemize}
		\item El usuario podrá hacer uso de su cuenta para acceder a los servicios de la biblioteca.
		\item Control de cada usuario registrado.
		\end{itemize}				
		}
		\UCitem{Autor(es)}{
		\begin{itemize}
		 \item Burciaga Ornelas Rodrigo Andrés
		 \item López Eder Guillermo
		\end{itemize}		 		 
		}
		\UCitem{Status}{Aprobado}
		\UCitem{Revisó}{
		\begin{itemize}
		\item Recoder Jiménez Bryan 
		\item Castillo Huitrón Sachiel 
		\end{itemize}				
		}
	\end{UseCase}
		%-------------------------------------- COMIENZA descripción Trayectoria Principal
	\begin{UCtrayectoria}{Principal}
	\UCpaso[\UCactor] Ingresa al sistema mediante el enlace a la aplicacion web
	 \UCpaso[\UCactor] Selecciona agregar usuario de la interfaz \IUref{UI01}{Gestión Usuario}.
	 \UCpaso[\UCsist]Muestra la interfaz \IUref{UI02}{Agregar Usuario}.
	  \UCpaso[\UCactor] Llena el formulario de registro
	   \UCpaso[\UCsist] Verifica la  \BRref{RN42}{Campos llenos usuario}.[Trayectoria A]
	    \UCpaso[\UCsist] Verifica la \BRref{RN39}{Formato del CURP}.[Trayectoria B]
	    \UCpaso[\UCsist] Se conecta a la base de datos. [Trayectoria C]
	    \UCpaso[\UCsist] Verifica la \BRref{RN43}{El usuario ha sido registrado anteriormente}.[Trayectoria D]
	    \UCpaso[\UCsist] Guarda toda la información ingresada en una instancia de usuario final.
	  \UCpaso[\UCsist]Envía la información del campo curp a la API de RENAPO.[Trayectoria E]
	  \UCpaso[\UCsist]Obtiene la información de la API de RENAPO. [Trayectoria F]
	  \UCpaso[\UCsist]El sistema valida la información ingresada por el usuario comparando con la enviada por la API. [Trayectoria G]
	    \UCpaso[\UCsist]Informa que el registro fue exitoso, guarda la instancia del usuario en la base de datos y envia un correo de confirmación al usuario con los siguientes datos: idUsuario (El id del usuario es generado automáticamente por la base de datos auto\_increment) y todos los campos de la instancia de usuario que ha sido almacenada en la base de datos.Se muestra en pantalla la \IUref{UI03}{Inserción Exitosa!}.
	    \UCpaso[\UCsist]Regresa al paso 2 de la trayectoria principal
	\end{UCtrayectoria}
			%-------------------------------------- COMIENZA descripción Trayectoria Alternativa.
		\begin{UCtrayectoriaA}{A}{Campos sin llenar.}
			\UCpaso[\UCsist] Enviara el  mensaje en pantalla \MSGref{MSG01}{Datos incompletos. Favor de completar todos los campos de entrada marcados con un *}
			\UCpaso[\UCsist] Regresa al paso 4 de la trayectoria principal conservando los datos de los campos que si fueron llenados con algún valor.
		\end{UCtrayectoriaA}
		
		
		\begin{UCtrayectoriaA}{B}{Los datos ingresados en el campo CURP no son correctos.}
			\UCpaso[\UCsist] Enviara el mensaje en pantalla \MSGref{MSG02}{Los datos ingresados en el campo CURP no son congruentes}
			\UCpaso[\UCsist] Señala mediante un mensaje en pantalla el tipo de formato que se debe seguir para validar el campo CURP.
			\UCpaso[\UCsist] Regresa al paso 4 de la trayectoria principal conservando los datos de los campos que si fueron llenados con algún valor y borrando el dato introducido en el campo CURP.		
		\end{UCtrayectoriaA}	
		
		
		\begin{UCtrayectoriaA}{C}{No se pudo conectar a la base de datos.}
			\UCpaso[\UCsist] Enviara un mensaje en pantalla   \MSGref{MSG03}{Problemas para acceder a la base de datos, intentelo más tarde}.
			\UCpaso[\UCsist] Envía un correo al administrador del sistema informando del error.
			\UCpaso[\UCsist] Se regresa al paso 2 de la trayectoria principal.
		\end{UCtrayectoriaA}		
		
		
		
		\begin{UCtrayectoriaA}{D}{Usuario existente en la base.}
			\UCpaso[\UCsist] Enviara el mensaje en pantalla  \MSGref{MSG04}{Usuario Registrado anteriormente, intenta con otro usuario}
			\UCpaso[\UCsist] Se regresa al paso 4 de la trayectoria principal eliminando todos los datos que han sido introducidos anteriormente.
		\end{UCtrayectoriaA}
		
		
		\begin{UCtrayectoriaA}{E}{La API no se encuentra disponible}
			\UCpaso[\UCsist] Guardará la información de la instancia en la base de datos.
			\UCpaso[\UCsist]Intentará enviar la información cada media hora hasta que este disponible la API de Renapo.
			\UCpaso[\UCsist] Enviara un mensaje en pantalla \MSGref{MSG05} {Validando datos, posteriormente se le informará el estado de su registro}
			\UCpaso[\UCsist] Se regresará al paso 2 de la trayectoria principal
			\UCpaso[\UCsist]  Enviará un correo al usuario y al administrador del sistema en caso de no poder establecer la comunicación con la API.
			\UCpaso[\UCsist] Se continuará al paso 11 en caso de éxito al establecer la comunicación con la API.
		\end{UCtrayectoriaA}
		
		\begin{UCtrayectoriaA}{F}{No se pudieron obtener los datos de la API}
			\UCpaso[\UCsist] Enviará el mensaje en pantalla  \MSGref{MSG06}{No se pudieron obtener los datos de la API de Renapo}
			\UCpaso[\UCsist] Se regresará al paso 2 de la trayectoria principal
		\end{UCtrayectoriaA}		
		
		
		
		\begin{UCtrayectoriaA}{G}{Los datos no coinciden}
			\UCpaso[\UCsist] Enviará el mensaje  \MSGref{MSG06}{Los datos ingresados en CURP no son correctos}
			\UCpaso[\UCsist] Se regresará al paso 4 de la trayectoria principal conservando todos los datos a excepción de la CURP.

		\end{UCtrayectoriaA}
		
		
		
		
		
		
%-------------------------------------- TERMINA descripción del caso de uso.