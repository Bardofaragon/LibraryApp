%%caso de uso 19 Modificar rol 
\begin{UseCase}{CU19}{Modificar rol}{
		El administrador podrá modificar el rol deseado de entre los que existen en la base de datos de acuerdo a los cargos de los trabajadores.
}
	\UCitem{Versión}{1.2}
	\UCitem{Actor}{Administrador.}
	\UCitem{Propósito}{El usuario puede modificar los roles dados de alta en el sistema ingresando el nombre actual y el nombre por el que desea cambiarlo.}
	\UCitem{Entradas}{Nombre actual del rol, nombre nuevo del rol}
	\UCitem{Salidas}{Rol acutalizado y mensaje
		MSGref{MSG01}{Operación exitosa}
		}
	\UCitem{Precondiciones}{Existe el usuario Administrador dado de alta en la base de datos y al menos un registro más de otro rol.}
	\UCitem{Postcondiciones}{Un registro es modificado en la tabla Rol de la base de datos del sistema.}
	\UCitem{Autores}{Guarneros Santana Víctor Hugo y Mújica Márquez Víctor Edgar.}
	\UCitem{Status}{En revisión}
	\UCitem{Responsable de revisión}{Esteban Martínez}
\end{UseCase}


%%trayectoria principal
\begin{UCtrayectoria}{Modificar rol}
	\UCpaso[\UCactor] El Administrador selecciona el botón \IUbutton{Modificar} de la pantalla \IUref{IU02}{Directorio de Roles}
	\UCpaso[\UCsist] Despliega la pantalla \IUref{IU19}{Modificar rol}
	\UCpaso[\UCactor] El administrador ingresa el nombre del rol que desea modificar.
	\UCpaso[\UCactor] El administrador ingresa el nuevo nombre del rol.
	\UCpaso[\UCactor] Selecciona el botón \IUbutton{Modificar} de la pantalla \IUref{IU19}{Modificar rol}
	\UCpaso[\UCsist] Verifica regla de negocio \BRref{RN01}{Duplicidad de roles} y modifica el rol,si el rol a modificar no existe muestra el mensaje %\MSGref{MSG02}{Registro inexistente} ,
	Registro inexistente, \Trayref{A}
	\UCpaso[\UCsist] Muestra mensaje %\MSGref{MSG01}{Operación exitosa}
	Operación exitosa
	\UCpaso[\UCsist] Regresa a la pantalla \IUref{IU02}{Directorio de Roles}
\end{UCtrayectoria}